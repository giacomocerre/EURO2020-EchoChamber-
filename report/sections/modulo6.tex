\section{Conclusioni}

    Sulla base dei risultati ottenuti dalle analisi condotte sulla rete illustrati nel capitolo \ref{openq} (par. \ref{subsection:echochamb} e \ref{when}), possiamo affermare che la struttura topologica della rete generi tendenzialmente due agglomerati di utenti con opinioni di segno opposto ($C_{u}$, par. \ref{subsection:class}), a conferma della presenza di camere d'eco nella discussione avvenuta su Twitter a proposito del gesto dell'inginocchiarsi a sostegno del movimento BLM durante il campionato UEFA Euro 2020. 
    
    Tali agglomerati hanno assunto caratteristiche e comportamenti peculiari evidenti nell'interazione con gli altri utenti, che appare più ristretta a utenti della stessa opinione tra i sostenitori del gesto, ma che si spinge verso posizioni più estreme nella parte contraria. Tale differenza emerge anche nella scelta degli \textit{hashtags}, ben più politicizzati nei tweet contrari al gesto.
    
    Infine, l'analisi congiunta delle figure \ref{Contourmaps} e \ref{hubs_evolution} ha evidenziato una possibile correlazione fra la crescita della rete, la comparsa degli utenti \textit{hubs} e la polarizzazione della rete. La formazione dell'\textit{echo chamber} nel nostro caso di studio, quindi, è sicuramente influenzata dall'algoritmo di \textit{feed} di Twitter \cite{Cinellie2023301118} -che ovviamente ha un ruolo fondamentale nel polarizzare l'opinione- , e potrebbe anche derivare dalla presenza nella rete di nodi con un alto valore di \textit{degree centrality}. Un hub diventa tale perché è stato ritwittato, citato o quotato da molte persone, le quali, collegandosi ad altri utenti, per lo più con opinioni simili (coefficiente di \textit{assortativity} > 0), talvolta altrettanto centrali, hanno avvicinato altre parti della rete della stessa fazione (fig. \ref{echo_chamber_clean}). 
    