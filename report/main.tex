\documentclass[9pt,twocolumn]{article}
\usepackage{inputenc, enumitem}
\usepackage[a4paper, top=3cm, bottom=3cm, left=2.5cm, right=2.5cm]{geometry}            % per rimpicciolire il margine superiore (ora è lo standard)
\setlength{\columnsep}{0.6cm}                % per rimpicciolire lo spazio tra le colonne
\newcommand\tab[1][1cm]{\hspace*{#1}}       % per il tab

\usepackage[table]{xcolor}
\definecolor{lightgray}{gray}{0.957}

\newcommand{\myquotation}[3]{
    \begin{list}{}{
      \vspace{0.5cm}
      \setlength\leftmargin{0.1\pdfpagewidth}
      \setlength\parindent{0cm}
   }
      \item\footnotesize{#3}
      \vspace{0cm}
      \item\hfill{\footnotesize{#1, \emph{#2}}}
   \end{list}
}

\usepackage{xcolor,listings}
\usepackage{textcomp}
\usepackage{color}

\definecolor{codegreen}{HTML}{3e7d7e}
\definecolor{codegray}{rgb}{0.5,0.5,0.5}
\definecolor{codeORANGE}{HTML}{008000}
\definecolor{codepurple}{HTML}{ba2121}
\definecolor{backcolour}{HTML}{f7f7f7}
\definecolor{bookColor}{cmyk}{0,0,0,0.90}  
\color{bookColor}

\lstset{upquote=true}
\usepackage{minted}
\usepackage{sourcecodepro}

\usepackage{letltxmacro}
% https://tex.stackexchange.com/q/88001/5764
\LetLtxMacro\oldttfamily\ttfamily
\DeclareRobustCommand{\ttfamily}{\oldttfamily\csname ttsize\endcsname}
\newcommand{\setttsize}[1]{\def\ttsize{#1}}%
\setttsize{\footnotesize}

\lstdefinestyle{mystyle}{
    backgroundcolor=\color{backcolour},   
    commentstyle=\color{codegreen},
    keywordstyle=\color{codeORANGE},
    numberstyle=\numberstyle,
    stringstyle=\color{codepurple},
    basicstyle=\setttsize{\footnotesize}\ttfamily,
    breakatwhitespace=true,
    breaklines=true,
    captionpos=b,
    %keepspaces=true,
    %numbers=left,
    %numbersep=0pt,
    %showspaces=false,
    showstringspaces=false,
    showtabs=false
}
\lstset{style=mystyle}

\usepackage{caption}
\DeclareCaptionFont{xxv}{\fontsize{7}{3.5}\selectfont}
\captionsetup{font=xxv}

\usepackage[italian]{babel}
\usepackage{colortbl}
\setlist[itemize]{noitemsep, nolistsep} 
\setlist[enumerate]{noitemsep, nolistsep} % per eliminare gli spazi dopo gli elenci puntati

\usepackage{graphicx}
\usepackage{stfloats}
\usepackage{natbib}
\usepackage{multicol} % enable multicolumn
\usepackage{import}

\usepackage{wrapfig}
\usepackage{animate}
\usepackage{siunitx}  
\DeclareSIUnit\permille{\text{\textperthousand}}
\usepackage{mathtools}
\usepackage{etoolbox}
\usepackage{booktabs}
\makeatletter
\usepackage{amsfonts} 
\usepackage{hyperref}
\hypersetup{
    colorlinks=true, 
    linkcolor=black,
    filecolor=blue,
    citecolor = black,      
    urlcolor=blue,
    }


%\usepackage{titlesec}                       % spazio tra le sezioni 
%\titlespacing*{\(sub)section}{spazio a sinistra}{spazio sopra il titolo}{spazio sotto la (sub)sezione}

%\titlespacing{\subsection}{0pt}{.5\baselineskip}{.3\baselineskip}
%\titlespacing{\section}{0pt}{1\baselineskip}{.5\baselineskip} 
\usepackage{caption}
\usepackage{subcaption}


\patchcmd{\chapter}{\if@openright\cleardoublepage\else\clearpage\fi}{}{}{}
\makeatother
\usepackage{multirow}

\begin{document}

\title{
%\vspace{1cm}
\textbf{Quando nasce un'echo chamber\\}
\vspace{0.4cm}
\large Il movimento \texttt{\#BlackLivesMatter} nel Campionato europeo di calcio 2021 \\ e il fenomeno \texttt{\#iononmiinginocchio}\\
\vspace{1.1cm}
}

\author{
    \and \and
    \textbf{Chiara Buongiovanni} \\ \small{c.buongiovanni@studenti.unipi.it} \\
    \small{507164}\\
    \and 
    \textbf{Roswita Candusso} \\ \small{r.candusso@studenti.unipi.it} \\
    \small{482938}\\
    \and \and \and
    \textbf{Giacomo Cerretini} \\ \small{g.cerretini2@studenti.unipi.it} \\
    \small{543999}\\
    \and 
    \textbf{Diego Febbe} \\
    \small{d.febbe@studenti.unipi.it}\\
    \small{536738} \\
}
\date{}
\begin{titlepage}
    \maketitle
    \vspace{0.5cm}
    \begin{abstract}
       In questa relazione viene presentato lo studio condotto sulla rete sociale reale creata a partire dai dati estratti dal social network Twitter durante il campionato di calcio \textbf{UEFA Euro 2020} inerenti al dibattito sorto intorno al gesto dell’inginocchiarsi a sostegno del movimento \textit{Black Lives Matter} prima della partita. 
    \end{abstract}
    
    \thispagestyle{empty}
        \vspace{3cm}
        \centering
        Relazione del progetto finale \\
        \Large
        Social Network Analysis \\
        \vspace{0.2cm}
        \date{\small A.A. 2020/2021} \\
        \vspace{1cm}
        \centering
        \includegraphics[scale=0.5]{logo2.png} 
\end{titlepage}


\maketitle


%\tableofcontents
%\include{esempio_modulo} da esempio


\import{sections/}{modulo1}
\import{sections/}{modulo2}
\import{sections/}{modulo3}
\import{sections/}{modulo4}
\import{sections/}{modulo5} 
\import{sections/}{modulo6} 

\bibliographystyle{plain} % We choose the "plain" reference style
\bibliography{references} 
%\bibliographystyle{unsrt}
%\bibliography{references}
\end{document}
